\section{Hardware Synthesis}\label{sec:abstraction}
%
Formal verification tools~\cite{abc,DBLP:conf/fmcad/BradleyM07,vis} for 
hardware typically synthesize an input design given in RTL into a netlist.  
This section briefly describes various netlist formats at -- \emph{bit-level},
\emph{word-level}, \emph{term-level} and \emph{software-level}.  
%
%
\subsection{Bit-level Synthesis}
A bit-level netlist is typically represented in \emph{AIG} (And-Inverter Graph), 
and stored in formats such as \emph{AIGER}, \emph{BLIF}, 
\emph{EDIF}, \emph{PLA} or \emph{BAF}.  This approach misses 
the opportunity to exploit the word-level structure of the 
input RTL design.  

\subsection{Word-level Synthesis}
A word-level netlist is typically represented in \emph{BTOR} or 
some word-level format that resembles SMT-LIB language~\footnote{http://smtlib.cs.uiowa.edu/}.  
This enables the use of word-level decision procedures, such as 
Satisfiability Modulo Theory (SMT) solvers, in the back-end of these tools.  

\subsection{Term-level Synthesis}
A term-level synthesis generates a term-level model represented 
in its custom language called counter arithmetic with lambda 
expressions and uninterpreted functions (CLU)~\cite{uclid}.  Tools 
like UCLID~\cite{uclid} uses term-level models, which are obtained 
from the word-level RTL description by data-path abstraction and 
abstracting functional blocks by uninterpreted functions.  The 
generated partially interpreted term-level abstract model is then 
solved using SMT solvers.  

\subsection{Software-level Synthesis}
Most recently, synthesis of RTL to a software program is presented 
in our previous works~\cite{mkm2015,mtk2016}.  This new flow enables application 
of classical software verification techniques like abstract interpretation, path-based 
symbolic execution and others, which has never been applied to RTL verification.  
In the subsequent section, we describe the synthesis of software netlist 
from hardware RTL in details.     
