\section{Verilog RTL to Software netlist}\label{sec:abstraction}
%
Formal verification tools~\cite{abc,DBLP:conf/fmcad/BradleyM07,vis} for 
hardware RTL designs typically synthesize the input RTL design into the
\emph{bit-level} or \emph{word-level} netlist.  
%This section briefly describes various netlist formats 
%at -- \emph{bit-level} or \emph{word-level}. 
%
%
More recently, synthesis of RTL into the software program is presented 
in our previous works~\cite{mkm2015,mtk2016}.  This new representation 
enables application of classical software verification techniques such as 
abstract interpretation, path-based symbolic execution and others, which 
have never been applied to RTL verification.  In the subsequent section, 
we describe the synthesis of software netlist from hardware RTL designs.     

Verilog and C share many common operators.  However, Verilog support few 
additional operators like part-select, bit-select from vectors, concatenation
operators, reduction operators which are not supported in C.  Additionally, Verilog 
statements like initial block, always block, generate statement, procedural assignment 
(blocking, non-blocking) and continuous assignment are 
not supported in C. Further, Verilog support 4-state valued data-types 
like reg, wire, integer.  All these non-trivial constructs combined with 
parallelism make Verilog to C translation challenging.

