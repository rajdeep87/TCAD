\section{Introductory Example} 

Inj this section, we present a simple example to give an overview our Verilog to C translation and to illustrate how properties for safety verification written in System Verilog Assertions (SVA) will also be translated also C code for our experiments. We assume the reader is familiar with the basics of Verilog and SVA~\cite{verilog}. \tmcmt{Rajdeep: what is the best reference?}

Fig.~\ref{fig:example} shows a Verilog module with a variety of typical constructs: register datatypes, initialization,
non-blocking and blocking assignments, an assign statement, and procedural blocks.  The synthesized hardware is the simple circuit shown in the centre.  On the right is the result of our translation of this Verilog RTL code into C.

The details of the translation are explained in Sec.~\ref{sec:v2c}. In summary, however, all the state-holding registers
in the Verilog design have been gathered into a C struct. A C function \texttt{top} is then defined that has the same paramaters as the Verilog module is represents. When invoked, this function executes a sequence of state updates that simulate a single clock-cycle of hardware execution, according to the Verilog synthesis semantics. Because this is a sequentialization of the parallel register updates modelled by the Verilog, some shadow variables are introduced that record register values at the start of the clock cycle for subsequent use. Note that the non-synthesizable Verilog initial block is not included here, as this is a testbench component. Note also that placement of the assignment to \texttt{c} at the end of the body of \texttt{top} follows the `read after write' semantics for Verilog combinational logic.

The \texttt{main()} procedure of 
the software netlist is shown on the right side of Figure~\ref{intro-fig3}, 
which also contains the initial block.
%  

We are interested in verifying the correctness of the Verilog RTL design 
against safety properties. 
%

Figure~\ref{intro-fig3} gives the safety properties that are checked 
against the RTL design of Figure~\ref{intro-fig1}.  The column on the 
left of Figure~\ref{intro-fig3} 
gives the SVA properties denoted by the property identifier 
$P0 \ldots P5$ and the column on the right gives the 
corresponding assertions in the software netlist.  The properties 
in Figure~\ref{intro-fig3} are of two types - global safety properties, 
and temporal properties containing the implication construct \texttt{(|->)} of SVA.  
The translation of SVA to assertions in the software netlist is described 
in detail in Section~\ref{prop}. 
%  
Intuitively, the \texttt{\#\#N} delay operator in SVA is modeled by invoking 
the top level procedure \texttt{top} in the software netlist $N$ times.  
Since we are interested in property verification of RTL designs by translating 
them to software netlist, so we consider the notion of equivalence between 
the Verilog RTL and the software netlist that preserves both the 
input-output behavior and the validity of all assertions of the RTL in the 
software netlist. 

\begin{figure*}[t]
\begin{center}
\small
\begin{tabular}{@{}lcl@{}} 
\hline\noalign{\vskip0.25ex}
\textbf{Verilog RTL} & \multicolumn{1}{l}{\textbf{Synthesized Hardware}} & \textbf{Software Netlist} \\
\hline
\begin{lstlisting}[mathescape=true,language=Verilog,basicstyle=\scriptsize\ttfamily]
module top(clk, a, c, out); 
input clk , a;
output [1:0] c;
output reg [3:0] out;
reg b,e; reg [1:0] d;
initial begin
  b=0;d=2'b0;
  e=0;out=4'b0;
end
assign c = e ? 1'b0 : d; 
always @(posedge clk) 
begin
 b<=a;
 if(b) e <= b; 
 else  e <= 0; 
 d <= e + 1;
end
always @(posedge clk) 
begin
  out <= d*d;
end  
endmodule
\end{lstlisting}
&
\begin{minipage}{4.3cm}
\centering
\scalebox{.6}{\import{figures/example/}{ckt.pspdftex}}
\end{minipage}
&
\begin{lstlisting}[mathescape=true,language=C,basicstyle=\scriptsize\ttfamily]
struct state_elements_top {
  unsigned int b,e;
  unsigned char d,out;
};
struct state_elements_top u1;

void top(unsigned int clk, 
 unsigned int a, 
 unsigned char *c, 
 unsigned char *out) {
 // shadow variables 
 unsigned int b_old = u1.b&0x1;
 unsigned char d_old = u1.d&0x3;
 unsigned int e_old = u1.e&0x1;
  
 u1.b = a;
 if(b_old) 
  u1.e = b_old&0x1;
 else
  u1.e = 0;
 u1.d = (e_old+1)&0x3;
 // update the output 
 u1.out = 
 (d_old&0x3) * (d_old&0x3);
 *out = u1.out&0xF;
 *c = u1.e ? 0 : (u1.d&0x3);
}
\end{lstlisting}\\
\hline
\end{tabular}
\caption{Translation of a Simple Circuit in Verilog HDL to a Software Netlist in C}\label{fig:example}
\end{center}
\end{figure*}



\begin{figure}[bth]
\small
\begin{center}
\begin{tabular}{@{}ll@{}}
\hline\noalign{\vskip0.25ex}
\textbf{System Verilog Assertions} & \textbf{C Translation} \\
\hline
\begin{lstlisting}[mathescape=true,language=Verilog,basicstyle=\scriptsize\ttfamily]
P0: assert property (d >= e);
P1: assert property ((a == 1) 
    |-> ##1 (b == 1));
P2: assert property ((a == 1) 
    |-> ##2 (e == 1));
P3: assert property ((a == 1) 
    |-> ##3 (d == 2));
P4: assert property ((a == 0) 
    |-> ##4 (out == 1));
P5: assert property ((a == 1) 
    |-> ##4 (out == 4));
\end{lstlisting}
&
\begin{lstlisting}[mathescape=true,language=C,basicstyle=\scriptsize\ttfamily]
int main() {
 unsigned int clk, a;
 unsigned char c, out;
 // initial block
 u1.b=0;u1.d=0;
 u1.e=0;u1.out=0;
 while(1) {
  P0:assert(u1.d>=u1.e);
  if(a==1) { 
   top(clk,a,&c,&out);
   P1: assert(u1.b==1);
   top(clk,a,&c,&out);
   P2: assert(u1.e==1);
   top(clk,a,&c,&out);
   P3: assert(u1.d==2);}
  //check output register
  if(a==1) {
   top(clk,a,&c,&out);
   P5: assert(u1.out==4);}
  else if(a==0) {
   top(clk,a,&c,&out);
   P4:assert(u1.out==1); } 
 }
}
\end{lstlisting}\\
\hline
\end{tabular}
	\caption{Property Specification in SVA and C}
\label{fig:sva}
\end{center}
\end{figure}

The waveform in Figure~\ref{intro-waveform} captures the input-output behavior of the RTL design. 
%The input-output behavior of the software netlist and the RTL in figure~\ref{intro-fig2} matches 
Using state-of-the-art verifiers, we prove that the result of all assertions 
in Figure~\ref{intro-fig3} are the same in the RTL and the software netlist design. 
%
For example, the properties \texttt{P2:assert(u1.d==1);} and \texttt{P3:assert(c==1);} 
hold only for the first cycle in both RTL and software netlist, but fails in the 
subsequent cycles in both the designs.  All other properties 
hold globally in both the designs, that is, these properties are $k$-inductive 
for the same value of $k$ in the RTL and the software netlist design.
%
\begin{figure} 
\begin{center}
  \includegraphics[width=\columnwidth]{figures/example/waveform1.eps}%
	\caption{Waveform showing the Input-Output behavior of RTL in Figure~\ref{intro-fig2}}
\label{intro-waveform}
\end{center}
\end{figure}
%
