\section{Proposed Verification Tool Flow}
%
\begin{figure}[t]
\centering
\vspace*{0.3cm}
\scalebox{.55}{\import{figures/}{new_flow.pspdftex}}
\caption{Tool flow for hardware property verification\label{fig:toolflow}}
\end{figure}
%
%===============================================================================
%\para{Software Verifiers at the heart of Hardware Verification}
%===============================================================================
%
Figure~\ref{fig:toolflow} shows the tool flow for hardware RTL 
verification at various levels -- 
\emph{bit-level}, \emph{word-level}, and \emph{software netlist-level}.  
The bottom flow of figure~\ref{fig:toolflow} shows the bit-level 
verification flow using \ABC.
\ABC does not support Verilog, so we use an open-source synthesis tool,
\yosys\footnote{http://www.clifford.at/yosys/} to translate Verilog
RTL to BLIF or AIGER, which is then passed to \ABC for verification. 
Given a design in Verilog RTL, \yosys bit-blasts it into a bit-level netlist 
along with the property and the resultant netlist is expressed in And-Inverter 
Graph.  The AIG is represented in AIGER format which is one of the standard 
and most prevelant format used by majority of verification tools.


Whereas, the middle flow of figure~\ref{fig:toolflow} shows the 
word-level verification using the tool \ebmc~\footnote{\scriptsize{www.cprover.org/hardware/ebmc}}. 
\ebmc supports IEEE 1364.1 Verilog 2005, and synthesize it to a bit-level 
netlist represented in AIGER or a word-level netlist represented in 
SMT-LIB2 format.  The top flow of figure~\ref{fig:toolflow} shows a
verification flow that synthesize a software netlist from RTL using 
the tool \emph{v2c}.  A wide range of representative software 
verification techniques are applied to determine the safety of the
software netlist models. In particular, we use $k$-induction~\cite{SSS00}
(implemented in the tools CBMC~\cite{cbmc.tacas:2004} and
2LS~\cite{kiki}), interpolation
(CPAChecker~\cite{DBLP:conf/cav/BeyerK11}, IMPARA~\cite{impara}
implementing the IMPACT algorithm~\cite{ken}), abstract interpretation
(Ast{\'r}ee~\cite{DBLP:conf/esop/CousotCFMMMR05}), IC3/PDR
(SeaHorn~\cite{DBLP:conf/cav/GurfinkelKKN15}) and automata-based 
trace abstraction (UltimateAutomizer~\cite{DBLP:conf/tacas/HeizmannDGLMSP16}).
%
%===============================================================================
\section{Experimental Results}
%===============================================================================
In this section, we report experimental results for \emph{unbounded} safety 
verification of hardware RTL.  Our experimental contributions are two folds.
%
\begin{enumerate}
  \item Compare off-the-shelf formal verification tools for RTL verification at 
    \emph{bit-level}, \emph{word-level}, and \emph{software netlist level}.  
    To this end, we compare state-of-the-art hardware model checking tools, such as 
    \emph{ABC 1.01} (winner of HWMCC'15) and 
    \ebmcv~\footnote{\scriptsize{www.cprover.org/hardware/ebmc}}, 
    with various software analyzers from SV-COMP 2016, such as 
    \emph{UltimateAutomizer 3292eade} (winner of SV-COMP'16), 
    \emph{CPAChecker 1.4}, 
    \emph{SeaHorn (revision 07666c810d)}, \emph{2LS 0.3.4}, 
    and a commercial abstract 
    interpretation based tool, \emph{Astr{\'e}e}.  
  
 \item  Compare various unbounded verification engines such as $k$-induction, 
    Interpolation, IC3/PDR and Abstraction Interpretation that are employed by 
    verification tools from hardware and software domains.
\end{enumerate}
%
Our experiments were performed on an Intel Xeon machine running at
3.07\,GHz.  We restricted the resources to 5 hours and 32\,GB RAM per
benchmark.  All our benchmarks in Verilog, software netlist models in 
ANSI-C, scripts for running \yosys, \ABC, \ebmcv, and other software 
verification tools are uploaded to a publicly accessible archive website.
\footnote{\scriptsize{http://www.cprover.org/hardware/tcad/}}
%
%===============================================================================
\para{Benchmarks}
%===============================================================================
%
We target two different classes of circuits: data-path intensive
circuits, including Huffman encoder/decoder and a Digital Audio
Input-Output chip (DAIO); and control-intensive designs,
including a non-pipelined 3-stage processor, a Read-Copy-update
mutual exclusion protocol, a FIFO controller, a buffer allocation model,
and an instruction queue controller.  The benchmarks in our paper 
are derived from real world hardware benchmark suites, including VIS 
Verilog models, the Texas-97 Benchmark suite, and opencores.org.
%
%-------------------------------------------------------------------------------
\para{Properties}
%-------------------------------------------------------------------------------
%
The safety properties are specified as System Verilog assertions (SVA).
The properties are instrumented as assertions in the software netlist. 
Our tool support fragment of SVA properties. Below, we present
an example of property monitor in software netlist corresponding 
to the concurrent assertions in SVA.  Let us consider verification 
of the following properties of Vending machine and Huffman 
encoder/decoder design. Figure~\ref{figure:prop1} and 
Figure~\ref{figure:prop2} shows the SVA property in the left and 
the corresponding monitor for the software netlist in the right. \\
$Assert\_1$: {\em The balance is never negative and never reaches 15.} 
\vspace{-3mm}
\begin{figure}[htbp]
\scriptsize
\begin{tabular}{l|l}
\hline
SVA & Monitor (in C)
\\
\hline
\begin{lstlisting}[mathescape=true,language=Verilog]
p1: assert property 
 (@(posedge clk) 
 vending.total[3]==0 && 
 !(vending.total[4:0]==15)); 
\end{lstlisting}
&
\begin{lstlisting}[mathescape=true,language=C]
int monitor_p1() {
 assert((((vending.total 
 >> 3) & 0x1) == 0) 
 && !(vending.total 
 & 0x1F == 15));
 return 1;
}
\end{lstlisting} \\
\hline
\end{tabular}
\caption{Modeling concurrent assertions in SVA as monitor in software netlist}
\label{figure:prop1}
\end{figure}
%

$Assert\_2$: {\em When a new transmission begins, the decoder is ready in the next clock.} 
\begin{figure}[htbp]
\scriptsize
\begin{tabular}{l|l}
\hline
SVA & Monitor (in C)
\\
\hline
\begin{lstlisting}[mathescape=true,language=Verilog]
p2: assert property 
 (@(posedge clk) 
 encoder.shiftreg[9:1] 
 == 1 |-> ##1 
 decoder.leaf == 1);
\end{lstlisting}
&
\begin{lstlisting}[mathescape=true,language=C]
int monitor_p2() {
 if(((encoder.shiftreg >> 1) 
 & 0x1FF) == 1) {
  check_consequent = 1;
  // call to top level 
  // module of design
  huffman(clk,addr);  
  if(check_consequent == 1) 
    assert(decoder.leaf == 1);
 }
}
\end{lstlisting} \\
\hline
\end{tabular}
\caption{Modeling temporal properties in SVA as monitor in software netlist}
\label{figure:prop2}
\end{figure}
%
%-------------------------------------------------------------------------------
\para{Discussion}
%-------------------------------------------------------------------------------
Figures~\ref{fig:kind}--\ref{fig:hybrid} report the comparison of various unbounded 
verification techniques employed by verification tools at bit-level, word-level, 
and software-level.  We categorize the approaches into three classes:
\begin{compactitem}
\item $k$-induction (Figure~\ref{fig:kind})
\item interpolation (Figure~\ref{fig:impact}), and 
\item PDR together with other hybrid techniques (Figure~\ref{fig:hybrid}).  
\end{compactitem}
By hybrid techniques, we refer to predicate
abstraction as implemented in \emph{CPAChecker} and a combination of
$k$-induction, BMC and abstract interpretation as implemented in
\emph{2LS}~\cite{kiki}.  On the $x$-axis is the analysis time in
seconds and on the $y$-axis we list the benchmarks. The vertical red lines on
the right-hand side of the diagrams show timeouts, out of memory,
inconclusive (unknown) results, errors (crashes), and wrong results
(tool bugs) reported by the tools. The tools can be distinguished 
by the size of the circles as well as by colour. 

\para{Analysis using $k$-induction} For safe benchmarks, the results
for bit-level, word-level verifiers and software verifiers are
comparable when the properties are 1-inductive or 2-step inductive.
However, for complex safety properties, \ABC and other abstraction
based software analyzers either timeout or took a long time to
terminate.  We investigated the reason for higher verification times
for some safe benchmarks, such as the FIFO controller, the RCU, and Buffer 
Allocation.  We observe that the properties are not $k$-inductive for
sufficiently large values of $k$, e.g.\ (k=1000) and thus tools based
on $k$-induction either timeout or took long time to
compute the inductive invariant sufficient to prove the property. For
the unsafe benchmarks, for example DAIO and the traffic light controller, where
the bugs are manifested only at 64 and 65 clock cycles respectively,
the verification times using \ABC and \ebmc's $k$-induction engine 
are comparable to \cbmcv and \textsc{2LS}. Figure~\ref{fig:kind} 
reports the time taken by the $k$-induction engine in 
\ABC, \ebmcv, \cbmcv and \textsc{2LS}.  We did not report
the time for \emph{CPAChecker} since the results suggest that 
its $k$-induction engine is not as mature yet. 

\para{Analysis using Interpolation} Figure~\ref{fig:impact} reports
the time taken by the interpolation engine in \ABC, \textsc{IMPARA}
and \emph{CPAChecker}. \ABC is the fastest in 9 out of 12
designs. However, it times out on three complex benchmarks, RCU, FIFO
and BufAl, whereas the software interpolation tool, \textsc{IMPARA},
which implements IMPACT algorithm solved three instances out of which
one is the complex FIFO design; yet \textsc{IMPARA} either timed out or
ran out of memory for the remaining designs.  \emph{CPAChecker} solved
5 out of 12 cases.  None of the interpolation engines was able to
prove RCU and BufAl.

\para{Analysis using Hybrid techniques} Figure~\ref{fig:hybrid}
reports the time taken by the IC3/PDR engine in \ABC, \emph{SeaHorn}
and other hybrid techniques as implemented in \emph{CPAChecker} and
\textsc{2LS}. \ABC~is the clear winner here; it is the only tool that
proves the FIFO and BufAl benchmarks safe within the given 5h timeout.
\emph{SeaHorn}'s PDR engine solves half of the benchmarks, but
produces false negatives on the other half due to limited support for
bitvectors. \textsc{2LS} successfully solved 8 benchmarks and times 
out on four benchmarks.  \emph{CPAChecker}'s predicate abstraction reliably
solves 7 benchmarks, but times out on two benchmarks and reports three
wrong results. Note that none of the tools was able to prove RCU.
%We do not report the results using Astr{\'e}e since it requires manual
%directives for data and control partitioning to avoid imprecision;
%nonetheless it generates many false alarms for safe benchmarks.

%
%%%%%%%%%%%%%%%%%%%%%%%%%%%%%%%%%%%%%%%%%%%%%%%%%%%%%%%%%%%%%%%%%%%%%%%%%%%%%%%%
\begin{figure}
%\scalebox{0.9}{
\begin{tikzpicture}[scale=0.9]
\small
\pgfplotstableread{
Benchmark	ABC-kind	EBMC-kind	CBMC-kind	2LS-kind
BufAl	15000	15000	15000	15000
DAIO	59.92	73.62	229.72	54
Dekker	0.17	15000	15000	15000
FIFOs	15000	15000	15000	15000
Heap	0.34	15000	15000	15000
Huffman	0.02	0.01	0.21	0.24
Ibuf	0.01	0.01	0.26	0.3
RCU	15000	15000	15000	15000
TicTacToe	0.01	0.01	0.23	0.25
n-pipe-mp	0.02	0.26	0.27	0.17
traffic-light	1.52	32.67	29.46	7.98
Vending	0.02	0.02	0.21	0.25
}\datatable

\begin{axis}[
    xbar, xmode=log,
    xmin=0,         
    ytick=data,    
    yticklabels from table={\datatable}{Benchmark},  
    legend style={at={(0.9,1.3)}},
]
\addplot [mark size=5pt,only marks, fill=yellow] table [x={ABC-kind}, y expr=\coordindex] {\datatable};    
\addplot [mark size=4pt,only marks, fill=green!70!blue]table [x={EBMC-kind}, y expr=\coordindex] {\datatable};
\addplot [mark size=3pt,only marks, fill=red!80!yellow] table [x={CBMC-kind}, y expr=\coordindex] {\datatable};
\addplot [mark size=2pt,only marks, fill=blue] table [x={2LS-kind}, y expr=\coordindex] {\datatable};
\addplot [red,sharp plot] coordinates{(15000,0) (15000,11)}
          node [left,rotate=90] at (axis cs:9000,10) {timeout};
\legend{{ABC-kind},{EBMC-kind},{CBMC-kind},{2LS-kind}}
\end{axis}
\end{tikzpicture}
%}
\caption{\label{fig:kind}
Comparison of $k$-induction tools
}
\end{figure}
%%%%%%%%%%%%%%%%%%%%%%%%%%%%%%%%%%%%%%%%%%%%%%%%%%%%%%%%%%%%%%%%%%%%%%%%%%%%%%%%

%%%%%%%%%%%%%%%%%%%%%%%%%%%%%%%%%%%%%%%%%%%%%%%%%%%%%%%%%%%%%%%%%%%%%%%%%%%%%%%%
\begin{figure}
%\scalebox{0.9}{
\begin{tikzpicture}[scale=0.9]
\small
\pgfplotstableread{
Benchmark	ABC-interpolation	CPA-interpolation	IMPARA-interpolation
BufAl	15000	60000	15000
DAIO	65.74	60000	240000
Dekker	3.73	240000	30000
FIFOs	15000	240000	0.39
Heap	1215.79	60000	15000
Huffman	0.01	3.87	0.41
Ibuf	0.01	4.04	30000
RCU	15000	60000	15000
TicTacToe	0.04	3.3	7870
non-pipe-mp	0.01	3.44	120000
traffic-light	5.79	240000	15000
Vending	0.01	3.65	120000
}\datatable

\begin{axis}[
    xbar, xmode=log,
    xmin=0,         
    ytick=data,    
    yticklabels from table={\datatable}{Benchmark},  
    legend style={at={(0.9,1.2)}},
]
\addplot [mark size=4pt,only marks, fill=yellow] table [x={ABC-interpolation}, y expr=\coordindex] {\datatable};    
\addplot [mark size=3pt,only marks, fill=orange]table [x={CPA-interpolation}, y expr=\coordindex] {\datatable};
\addplot [mark size=2pt,only marks, fill=gray] table [x={IMPARA-interpolation}, y expr=\coordindex] {\datatable};
\addplot [red,sharp plot] coordinates{(15000,0) (15000,11)}
          node [left,rotate=90] at (axis cs:9000,11) {timeout};
\addplot [red,sharp plot] coordinates{(30000,0) (30000,11)}
          node [left,rotate=90] at (axis cs:19000,11) {memout};
\addplot [red,sharp plot] coordinates{(60000,0) (60000,11)}
          node [left,rotate=90] at (axis cs:40000,11) {unknown};
\addplot [red,sharp plot] coordinates{(120000,0) (120000,11)}
          node [left,rotate=90] at (axis cs:80000,11) {error};
\addplot [red,sharp plot] coordinates{(240000,0) (240000,11)}
          node [left,rotate=90] at (axis cs:180000,11) {wrong};
\legend{{ABC-interpolation},{CPA-interpolation},{IMPARA}}
\end{axis}
\end{tikzpicture}
%}
\caption{\label{fig:impact}
Comparison of interpolation-based tools
}
\end{figure}
%%%%%%%%%%%%%%%%%%%%%%%%%%%%%%%%%%%%%%%%%%%%%%%%%%%%%%%%%%%%%%%%%%%%%%%%%%%%%%%%

%%%%%%%%%%%%%%%%%%%%%%%%%%%%%%%%%%%%%%%%%%%%%%%%%%%%%%%%%%%%%%%%%%%%%%%%%%%%%%%%
\begin{figure}
%\scalebox{0.9}{
\begin{tikzpicture}[scale=0.9]
\small
\pgfplotstableread{
Benchmark	ABC-pdr	SeaHorn-pdr	CPA-predabs	2LS-kiki
BufAl	12250.5	240000	240000	15000
DAIO	0.03	2.16	3.95	55.72
Dekker	0.03	4.83	240000	15000
FIFOs	3759.75	240000	15000	15000
Heap	0.75	240000	10.09	0.91
Huffman	0.02	0.09	4.01	0.25
Ibuf	0.01	240000	3.95	0.73
RCU	15000	240000	15000	15000
TicTacToe	0.03	240000	3.29	0.21
non-pipe-mp	0.01	0.06	3.46	0.20
traffic-light	0.13	1.44	240000	7.48
Vending	0.02	1.28	3.77	0.34
}\datatable

\begin{axis}[
    xbar, xmode=log,
    xmin=0,         
    ytick=data,    
    yticklabels from table={\datatable}{Benchmark},  
    legend style={at={(0.9,1.3)}},
]
\addplot [mark size=5pt,only marks, fill=yellow] table [x={ABC-pdr}, y expr=\coordindex] {\datatable};    
\addplot [mark size=4pt,only marks, fill=green!70!blue]table [x={SeaHorn-pdr}, y expr=\coordindex] {\datatable};
\addplot [mark size=3pt,only marks, fill=orange] table [x={CPA-predabs}, y expr=\coordindex] {\datatable};
\addplot [mark size=2pt,only marks, fill=blue] table [x={2LS-kiki}, y expr=\coordindex] {\datatable};
\addplot [red,sharp plot] coordinates{(15000,0) (15000,11)}
          node [left,rotate=90] at (axis cs:9000,11) {timeout};
\addplot [red,sharp plot] coordinates{(30000,0) (30000,11)}
          node [left,rotate=90] at (axis cs:19000,11) {memout};
\addplot [red,sharp plot] coordinates{(60000,0) (60000,11)}
          node [left,rotate=90] at (axis cs:40000,11) {unknown};
\addplot [red,sharp plot] coordinates{(120000,0) (120000,11)}
          node [left,rotate=90] at (axis cs:80000,11) {error};
\addplot [red,sharp plot] coordinates{(240000,0) (240000,11)}
          node [left,rotate=90] at (axis cs:180000,11) {wrong};
\legend{{ABC-pdr},{SeaHorn-pdr},{CPA-predabs},{2LS-kiki}}
\end{axis}
\end{tikzpicture}
%}
\caption{\label{fig:hybrid}
Comparison of hybrid techniques
}
\end{figure}
%%%%%%%%%%%%%%%%%%% Plot runtimes %%%%%%%%%%%%%%%%%%%%%%%
\begin{figure}[t]
  \centering
  \begin{tikzpicture}[scale=0.60]

\pgfplotscreateplotcyclelist{markstyles}{%
solid, every mark/.append style={solid, fill=white}, mark=square*, mark size=2.5\\%
solid, every mark/.append style={solid, blue}, mark=triangle*,mark size=2.5\\%
solid, every mark/.append style={solid, fill=black}, mark=otimes*,, mark
    size=2.5\\%
}
    
 	%axis
  \begin{axis}[
    width=\linewidth,
    xlabel={Benchmark Number},
    ylabel={Time (seconds)},
    domain = 1:40,
    xmin=1, xmax=34,
    ymin=0, ymax=200,
    %ytick={0,20,...,200},
    xtick={1,5,10,...,34},
    width=15cm, height= 7cm,
    ymode = log,
    %log basis x={2},
    %xticklabel=\pgfmathparse{2^\tick}\pgfmathprintnumber{\pgfmathresult},
    legend pos = north west,
    grid = major,
    major grid style={line width=.2pt,draw=gray!50},
    cycle list name=markstyles
  ]
	
  %plots
  %\addplot table [only marks, y=Time, x=Benchmarks]{plotdata/cbmc.dat};
	%\addlegendentry{CBMC}
  
  \addplot table [only marks, y=Time, x=Benchmark]{plotdata/ultimate.dat};
  \addlegendentry{UltimateAutomizer}
  
  \addplot table [only marks, y=Time, x=Benchmark]{plotdata/astree.dat};
  \addlegendentry{Astr{\'e}e}
	
  \end{axis}  
\end{tikzpicture}
\caption{\label{fig:runtimes}
  Runtime Comparison between UltimateAutomizer and Astr{\'e}e}
\end{figure}

%%%%%%%%%%%%%%%%%%%%%%%%%%%%%%%%%%%%%%%%%%%%%%%%%%%%%%%%%
\para{Analysis of Abstraction based tools}
%
Figure~\ref{fig:runtimes} gives the runtime comparsion between Ultimate Automizer 
and Astr{\'e}e.  Both tools operate on an over-approximate abstraction of the concrete
program.  We now discuss the analysis using these abstraction based tools.


%%%%%%%%%%%%%%%%%%%%%%%%%%%%%%%%%%%%%%%%%%%%%%%%%%%%%%%%%%%%%%%%%%%%%%%%%%%%%%%%
\para{Effect of Trace partitioning for Precise analysis in Astr{\'e}e}
%%%%%%%%%%%%%%%%%%%%%%%%%%%%%%%%%%%%%%%%%%%%%%%%%%%%%%%%%%%%%%%%%%%%%%%%%%%%%%%%
%
\para{Sources of Imprecision}  There can be various sources of imprecision in abstract 
interpretation.  Few common sources of imprecision may occur due to   
\emph{control-flow join}, \emph{loop widening} or use of \emph{imprecise abstract domain}.  
We analyzed the structure of software netlist models to detect the potential sources of 
imprecision that may occur during the analysis using Astr{\'e}e.  Recall that the software 
netlist model retains the control structure of the input RTL design.  Hence, if the original 
RTL has loops or conditional branches inside a module, a software netlist also preserves the 
similar control structure.   

\Omit{Standard Static Analysis is Imprecise, but can we do better than bit
blasting ? 

\para{Idea} Partition the traces so that we can prove correctness for each partition.
\para{Solution} To be effecient, we want partitions that are just precise
enough. 
}
\Omit{Best HW runtime versus Best Software runtimes}

\para{Summary of the results} We investigated the reason for large
number of timeouts, wrong results and errors produced by the software
verifiers.  We observed that software netlists heavily use bit-level 
operations and thus bit-precise reasoning ability is necessary for 
the underlying verification engine. However, bit-level operations 
are less prevalent in conventional software and hence less tested 
in software analysis tools. Also, software verification tools 
often use numerical abstractions, which are likely to lose 
important bit-precise information.  As a consequence, 
our results show that running conventional sotware verification 
tools on software netlists exhibits many tool bugs (``wrong'').

The abstraction and invariant inference techniques employed in
software tools such as \emph{CPAChecker} and \emph{2LS} have never
been optimized for hardware analysis. But the results in this paper
show that these tools are within one order of magnitude compared to
hardware model checkers for detecting bugs or proving safety for some
of the software netlist models.  We thus believe that there is scope
here for new tools that implement abstract interpretation using
abstract domains developed specifically for this task, e.g.~by
applying abstract conflict driven learning~\cite{dhk2013-popl}.

%-------------------------------------------------------------------------------
\para{Limitations of the Result}
%-------------------------------------------------------------------------------
Due to lack or absence of Verilog support in other open-source tools that
participate in the HWMCC competition, like \emph{PdTrav}, \emph{IIMC} and 
\emph{V3}, we could not run these tools on our benchmark suite. 


%===============================================================================
\section{Conclusions}
%===============================================================================
We present an approach to unbounded safety verification for hardware
designs given in Verilog RTL, at the heart of which are software
verifiers.  We present a comparison of unbounded verification
techniques at bit-level, word-level and \emph{software netlist} level.

\Omit{
The range of software verification techniques is vast; this paper can
thus only be an initial step.  We will evaluate further software
verification techniques and their application to hardware property
checking and co-verification workloads as future work.
}
