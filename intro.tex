\section{Introduction}\label{sec:intro}
%
Formal verification of hardware is, today, a well-established design technology~\cite{Seligman:2015:FV}. From its modest beginnings in the 1980s, extensive corporate and academic research into hardware verification has gone hand-in-hand with gradual but sustained industrial take-up. Two landmark achievements, by teams at Intel Corporation, were the full correctness verification of the entire execution clusters of the Intel Core 2 Duo~\cite{Core2} and Core i7 processors~\cite{i7}. The more widely-adopted methodology,
assertion-based verification~\cite{Foster:2009:AAB}, is by now well supported by mature software tools from the EDA companies.

The industrial success of formal hardware verification has been advanced by decades of research into specialised data structures and algorithms~\cite{vis}, tools~\cite{Seger:2005:IEE,abc}, and methodology~\cite{MCMILLAN2000279,Aagaard:2000:MLH}. The most-used technology relies on back-end analysis by propositional satisfiability (SAT) solvers~\cite{Biere1999} or by modern solvers for satisfiability modulo theories (SMT)~\cite{decision_procedures, DBLP:conf/lpar/AndrausLS08,soc-keating,
DBLP:conf/mtv/SunkariCVM07,DBLP:conf/cav/Bjesse08}. Scientifically inspired and practically fruitful research into hardware verification continues today, with scalability an ever-present challenge---along with many others. 

But the past two decades have also seen an explosion in research into formal software verfication~\cite{dkw2008}, with progress vivdly witnessed by the yearly `SV-COMP' competition~\cite{Beyer2017}.  Verification of industrial-scale software is now an established possibility, and commerial offerings are starting to appear. Some techniques used in software verification, such as interpolation~\cite{}, have analogues in hardware verification. Others, such as abstract interpretation~\cite{}, have been largely confined to the software domain.  

In this paper, we explore the potential for hardware verification to leverage and exploit past and future progress in software verification. We translate hardware models, articulated in Verilog at register transfer level, into 
a \emph{software netlist}---an ANSI-C program that faithfully reproduces the hardware in software. This opens the door to the exploring the application of software verification technologies, arising from contemporary program analysis research, to formal verification of hardware RTL designs. Our aim is to establish some baseline experimental data for verification of hardware by translation into software, and also to evaluate the effectiveness of some current software verification methods---notably abstract interpretation---which have not yet been well investigated for hardware.

Of course, the idea of expressing RTL designs in software has been 
advocated in the past, primarily to enable faster simulation; we 
mention only~\cite{soc-keating}, which highlights property 
verification-related benefits in the context of SoC design.  
But we emphasize that the software models of hardware used 
in~\cite{soc-keating} are abstractions of hardware which are 
usually written manually---and often disconnected from a `golden' RTL design model from which the chip 
is ultimately realized.  By contrast, our software netlist representation 
is not an abstract simulation model but an exact representation 
of the RTL. 

In particular, we present a novel property verification framework 
for hardware RTL designs in software language using native 
software analyzers.  To this end, we explore software verification 
technologies that use SAT/SMT decision procedures~\cite{DBLP:conf/cav/BeyerK11,
2ls,cbmc.tacas:2004,DBLP:conf/tacas/HeizmannDGLMSP16} as well as abstraction-based 
techniques such as abstract interpretation~\cite{CC79} for formal verification 
of RTL designs.  It is worth emphasizing that some of these techniques 
have never been applied to hardware verification.  
%

By means of experiment, we show that a commercial abstract interpreter  
Astr{\'e}e with manual guidance is effective for finding complex bugs 
as well as proving unbounded safety of the software netlist designs 
generated from the RTL circuits.  The performance of Astr{\'e}e is 
comparable to a bit-level hardware model checker \ABC~\cite{abc}.  
%
In some cases, Astr{\'e}e is faster than \ABC for finding bugs. 
%
However, Astr{\'e}e shows a high degree of imprecision on our benchmarks 
which is handled by manually guiding the analysis using various trace 
partitioning directives~\cite{DBLP:journals/toplas/RivalM07}.  
%
