\section{Introduction}\label{sec:intro}
%
Formal verification is now a well-established technology for in hardware design~\cite{Seligman:2015:FV}.  
From modest beginnings in the 1980s,
extensive corporate and academic research in hardware verification has gone
hand-in-hand with gradual but sustained industrial take-up.  Landmark
achievements in full correctness of commercial designs include the
verification of the entire execution clusters of the Intel Core 2
Duo~\cite{Core2} and Core i7 processors~\cite{i7} and the end-to-end ISA
verification of several ARM processors~\cite{ARM}.  The most prevalent
methodology, assertion-based verification~\cite{Foster:2009:AAB}, is now
well supported by mature tools supplied by electronics design automation (EDA) companies.

The industrial success of formal hardware verification has been enabled by
decades of research into specialised data structures and
algorithms~\cite{ic3, fmcad2000, ken, biere, STE},
tools~\cite{Seger:2005:IEE, abc, ebmc, vis, cadence, synopsysfv}, and
methodology~\cite{MCMILLAN2000279, Aagaard:2000:MLH, uclid, word-term,
word-bmc, DBLP:conf/lpar/AndrausLS08}.  The most widely used technology
relies on back-end analysis by propositional satisfiability (SAT)
solvers~\cite{Biere1999} or by modern solvers for satisfiability modulo
theories (SMT)~\cite{decision_procedures,
DBLP:conf/lpar/AndrausLS08,soc-keating,
DBLP:conf/mtv/SunkariCVM07,DBLP:conf/cav/Bjesse08}.  Scientifically inspired
and practically fruitful research into hardware verification continues
today, with scalability an ever-present challenge.

But the past two decades have also seen an explosion in research into
automated formal verfication of \emph{software}~\cite{dkw2008}, with
progress vivdly witnessed by the yearly `SV-COMP'
competition~\cite{Beyer2017}.  Verification of industrial-scale software is
now an established possibility, and commerial offerings are starting to
appear.  Some techniques used in software verification, such as
interpolation~\cite{Interpolants, Kroening:2011:ISV}, have analogues in
hardware verification.  Others, such as abstract
interpretation~\cite{CousotCousot77, Cousot:1996:AI}, have been largely
confined to the software domain.

In this paper, we explore the potential to leverage past and future progress
in automated software verification for hardware verification.  We~translate
a hardware model, articulated in Verilog at register transfer level (RTL),
into a \emph{software netlist}, an ANSI-C program that faithfully reproduces
the hardware in software.  This opens the door to experiments with software
verification technologies, arising from contemporary program analysis
research, to formal verification of hardware RTL designs.  We have two aims:
first, to establish baseline experimental data for verification of hardware
by translation into software; second, to evaluate the effectiveness of
software verification methods---notably abstract interpretation---that have
not yet been investigated for hardware.

Of course, the idea of expressing RTL designs in software has been
advocated in the past, primarily to enable faster simulation.  We mention
only~\cite{soc-keating}, which highlights verification-related benefits in
the context of SoC design.  But we emphasize that the software models
in~\cite{soc-keating} are abstractions of hardware that are usually written
manually and disconnected from the `golden' RTL design from which the chip
is ultimately realized.

By contrast, our software netlist representation is an exact
translation of the RTL design, not an abstract simulation model.  This is a
deliberate design choice, aiming to produce a faithful and transparent
software model that leaves maximum flexibility for software verification
proof strategies.  The principles behind the translation, which we have
implemented in a tool called V2C, are explained in Section~\ref{sec:v2c}.

Using our translation, we have experimentally investigated formal
verification of hardware using a range of native software analyzers.  For
baseline comparison with RTL verification tools, we experiment with software
verification technologies that employ SAT/SMT decision
procedures~\cite{DBLP:conf/cav/BeyerK11, 2ls, cbmc.tacas:2004,
DBLP:conf/tacas/HeizmannDGLMSP16}.  To probe the frontiers, we evaluate the
use of abstraction-based software verification techniques---most notably
abstract interpretation.  To ensure our results are openly accessible and
can be independently validated, our experiments compare the performance of
tools drawn from the Hardware Model Checking Competition
(HWMCC) against software verifiers from the Software Verification
Competition (SV-COMP).

Extensive experiments with 34 RTL benchmarks confirm that SAT-based,
bit-level hardware model checkers currently outperform bit-level software
analyzers.  But our experiments also show that a commercial abstract
interpreter, Astr{\'e}e, is---with manual guidance---particularly effective for
finding complex bugs, as well as for proving unbounded safety of software netlists.  
Unlike conventional SAT/SMT-based hardware verification, which
bit-blasts the design into propositional logic, abstract interpretation
analyzes the design over an abstract domain by computing a least fixed
point of the set of equations derived from the design.  Abstract
interpretation leverages the expressiveness of the underlying
numerical abstract domains for representing the relational and
non-relational invariants required for unbounded proofs or finding bugs.

Our experiments show that the performance of Astr{\'e}e is comparable to the
bit-level hardware model checker ABC~\cite{abc} and in some cases Astr{\'e}e
is faster for finding bugs.  It is worth noting that this technique does not
bit-blast the design, hence it is also more scalable than SAT/SMT-based
analysis.  However, abstract interpretation typically suffers from
imprecision due to the use of non-disjunctive abstract domains~\cite{nd}
that cannot precisely represent disjunctions.  Astr{\'e}e shows a high
degree of imprecision on our benchmarks.  We mitigate the imprecision by
manual trace partitioning, as explained in Section~\ref{abstraction}.

% ---------------------------------------------------------------------
% New Contributions Statement 
% ---------------------------------------------------------------------

Our previous work produced preliminary results 
for bounded and unbounded safety verification of RTL designs using 
software analyzers~\cite{mkm2015,mskm2016}.  
The results in these papers were restricted to a smaller set 
of benchmarks and fewer application of verification tools than we
explore here.  
The work presented here also makes significant technical contributions 
beyond those in~\cite{mkm2015,mskm2016} for formal 
verification of hardware RTL designs using state-of-the-art software analyzers 
based on SAT/SMT solvers as well as abstraction-based techniques. 
In particular, we present here a formal description of unbounded 
verification algorithms with a special emphasis on the application of  
classical abstract interpretation for precise safety verification.  In addition, 
this paper gives a more detailed analysis of a range of precise SAT/SMT-based 
and abstraction-based software verification tools for verifying a diverse set of 34 RTL 
circuits drawn from various sources.  We also present a notion of 
equivalence between a RTL and software netlist design for the purposes of RTL 
verification.  Finally, this paper reports on an experimental comparison of
bit-level and word-level verification using SAT and SMT solvers respectively. 

