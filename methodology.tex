\section{Methodology}~\label{methodology}
%===============================================================================
\subsection{Verification Techniques}
%===============================================================================

Given a transition system~$T$ and a property~$P$, we formally desribe 
various post-condition computation techniques employed by
model checkers based on SAT or BDDs.  The goal of post-condition computation
is to derive the reachable sets of states, which is the strongest possible
strengthening of $P$.  Traditional model checkers explicitly compute the
post-conditions or pre-conditions to compute the reachable sets of states or
the set of states that never lead to the violation of $P$ respectively. 
Here, we describe various algorithms for computing the reachable state space
of $T$ such that every reachable state satisfies the state property $P$. We 
broadly classify these techniques into two separate classes based on their 
usage in either hardware or software domain -- \emph{netlist verification}, 
\emph{software verification}, respectively. 

\subsection{Netlist Verification}
%
%-------------------------------------------------------------------------------
\para{Bounded Model Checking} 
%
The idea of BMC is as follows. 
Given a depth~$k$ and a set of error states~$B$, BMC operates
by unwinding the transition relation~$T$ up to depth~$k$ starting from
initial state $x_0$, represented by an initial state predicate~$I$.
This results in the following formula which is then checked for
satisfiability using an efficient SAT or SMT procedure.
%
\[ I(x_0) \wedge \delta(x_0, x_1) \wedge \ldots 
   \wedge \delta(x_{k-1}, x_k) \wedge (B(x_0) \vee \ldots \vee B(x_k)) \]
%
Thus, BMC exploits the finiteness of $T$ and creates k copies of the
$T$ for each unrolling.

%-------------------------------------------------------------------------------
\para{BMC with K-induction} 
%
In practice, BMC technique is incomplete. $k$-induction 
strengthens the induction by assuming that $P$ holds 
over $k-1$ time steps to increase the likelihood that $P$
holds in $k$-th time step. The base case of the $k$-induction is the
simple BMC problem shown before. If the base case is unsatisfiable,
then the induction step is checked which is shown below.
%
\[ P(x_0) \wedge_{i=0}^{k-1} (\delta_i \wedge P_i) \implies P_k \]
%
%-------------------------------------------------------------------------------
\para{Interpolation-based Model Checking} 
%
The interpolation-based model checking technique
computes a step-wise over-approximation of the reachable state-space,
$Q$, for a fixed unrolling $k$ of $T$, which is shown as
follows.
%
\[ Q \wedge_{i=0}^{k-1} (\delta_i \wedge P_i) \implies P_k \]
%
The value of $k$ is increased when the implication fails. This
progressively yields better over-approximations $Q_i$ thereby
generating an interpolant between the $i$-th over-approximation and
$k$-step unrolling which may be sufficient to prove the property
$P$. Thus, interpolation is a complete method for reachability
analysis of finite-state systems. In summary, the interpolation
technique computes an abstraction of the post-condition with respect
to the property $P$, by deriving interpolants from the failure of the
BMC problems.

%-------------------------------------------------------------------------------
\para{Predicate Abstraction with CEGAR}
%
The technique computes abstract models $\hat{T}$ from the concrete model
$T$ using existential abstraction. Given a concrete state $x$ which is the 
valuation of all registers and a set of predicates, $B=\{\pi_{i} \ldots \pi_{k}\}$, 
an abstract state $b$ is a vector of Boolean values which is obtained by applying 
all the predicates to the concrete state, denoted $b=\alpha(x)$, where $\alpha$
is an abstraction function. We define the abstract machine and property verification
in predicate abstraction as follows:
\begin{enumerate}
\item Transition Relation: $ \hat{\delta} := \{(b,b') | \exists x,x' \in S: \delta(x,x') 
\wedge \alpha(x) = b \wedge \alpha(x') = b' \} $, $x=\{x_1 \ldots x_n\}, 
b=\{b_1 \ldots b_n\}, b_i=\pi_{i}(x) $, $\pi_i$ is the predicate on concrete 
variable $x_i$

\item Initial State: $\hat{I}(b) := \exists x \in S: (\alpha(x) = b) \wedge I(x)$

\item Safety Property: $\hat{P}(b) := \forall x \in S: (\alpha(x) = b) \implies P(x)$ 
Thus, if property $\hat{P}$ holds on all reachable states of the abstract model $\hat{T}$,
then $P$ also holds on all reachable state of concrete model $T$.
\end{enumerate}

%-------------------------------------------------------------------------------
\para{Property Directed Reachability} 
%
Let $\alpha_1(x), \alpha_2(x),...,\alpha_n(x)$ be a sequence of 
inductive assertions generated by the algorithm such that if the 
following holds,
\begin{enumerate}
 \item $I(x) \implies P$
 \item $\forall i, I(x) \implies \alpha_i(x)$
 \item $\forall i, \wedge_{i=1}^{k} \alpha_i(x) \wedge P(x) \wedge \delta(x,x') \implies \alpha_k(x')$,
 $\alpha_k$ is inductive relative to $\alpha_1, \alpha_2, ...,\alpha_{k-1}$.
 \item $\forall i, \wedge_{i=1}^{n} \alpha_i(x) \wedge P(x) \wedge \delta(x,x') \implies P(x')$, \\
 $P$ is inductive related to the inductive invariants $\alpha_1, \alpha_2, ...,\alpha_n$.
\end{enumerate}
then $P$ is an invariant of $T$. The algorithm computes successive intermediate 
inductive assertions by removing the counterexamples-to-induction states in 
property-directed fashion.  Thus the technique computes an over-approximation 
of the set of reachable states in successive steps until it finds an 
inductive strengthening assertion sufficient to prove the property. 


\subsection{Program Verification}
%
We now describe some well known software verification techniques that are 
used in this paper for verifying software netlist models.  We refer software 
netlist as a program in the following description. 
  
%-------------------------------------------------------------------------------
\para{Abstract Interpretation} 
%
In abstract interpretation~\cite{DBLP:journals/corr/abs-cs-0701193,DBLP:conf/emsoft/Cousot07},
a given program is analyzed with respect to a given \emph{abstract domain}.  
The technique computes an abstraction of the post-condition to derive an 
inductive invariant $AInv$ that includes the start state and thus are 
over-approximations of the set of reachable states. Elements of an abstract 
domain can be sets or conjuncts of formulae~\cite{vmcai-2013}. So abstract 
interpretation can be viewed as:
\begin{enumerate}
\item  $\exists AInv \in A. \forall x_0, x_1. (I(x_0) \implies AInv(x_0)) 
\wedge (AInv(x_0) \wedge \delta(x_0, x_1) \implies AInv(x_1))$, 
\item $\forall x. AInv(x) \implies \neg{P(x)}$
\end{enumerate}
where $A$ is the set of formulae which are the elements of the 
chosen abstract domain. The system is safe if there is no model. 
Otherwise, the technique computes either a stronger invariant 
$AInv$ or choose a more expressive abstract domain to prove safety.

%-------------------------------------------------------------------------------
\para{Path-based symbolic execution}
%-------------------------------------------------------------------------------
%
Path-based symbolic execution is an well known technique in program verification 
that is used for automatic test case generation~\cite{DBLP:conf/osdi/CadarDE08}.
Techniques based on precise forward symbolic 
execution~\cite{DBLP:conf/osdi/CadarDE08, King:1976:SEP:360248.360252}
executes a single path of a program at a time with symbolic inputs 
to generate symbolic expressions which are then conjoined with
the specification and are checked using a SAT/SMT solvers. 
The generated symbolic expression contains logical conjunction 
of the guards and assignments along that path.  This approach generates many 
SAT/SMT queries and may suffer from path explosion problem.

A similar technique is proposed for RTL in~\cite{star}. 
For sequential circuits, symbolic expressions contains path variables which 
are annotated by the time cycle to which it belongs. The 
work of~\cite{DBLP:journals/todaes/LiuV14} analyzes 
feasibility of every path in the RTL program using a hybrid 
of concrete and symbolic execution and property based pruning.
%
%-------------------------------------------------------------------------------
\para{Automata based Trace Abstraction with Interpolants}
%-------------------------------------------------------------------------------
%
The automata-theoretic approach to program verification~\cite{DBLP:conf/cav/HeizmannHP13} construct 
an automaton to represent program traces along with the negation of the assertion and checks if the 
language of the automaton is empty. One of the automata-theoretic approach for program verification 
is implemented in a tool called \emph{UAutomizer}~\cite{DBLP:conf/tacas/HeizmannDGLMSP16}. 
It obtains a proof for a single trace and trace abstraction generalizes 
these proofs and combines them.  Each iteration of the 
algorithm tries to obtain loop invariants for the path program
induced by the current counterexample.  This may be done through different techniques 
which can be applied iteratively. Each technique provides a set of Hoare triples that constitutes a proof of
safety for the current counterexample.
As soon as a technique yields loop invariants, the algorithm uses these loop
invariants during the refinement and generalization step of trace abstraction.  If all the 
techniques have been exhausted and a loop invariant has not been obtained yet, the algorithm
use some or all (depending on the strategy) of the obtained hoare triples and 
refine or generalize with them.



