\section{Definition and Problem Statement}~\label{problem}
%
We formally define the safety verification of a transition system~$T$,
commonly known as \emph{hardware model checking problem}.  We define a
Boolean \textit{transition system}, appropriate for bit-level hardware
modeling, as $T = \langle I,x,\delta \rangle$, where $x = x_1,x_2,...,x_n$
is the set of program variables, which range over $\mathbb{B} =
\{\mathit{true}, \mathit{false}\}$.  A \emph{state} $s$ of $T$ is an
assignment of values to variables $x$, the predicate $I(x)$ is a formula over
the variables $x$ specifying the \emph{initial state} and $\delta(x,x')$
represents the transition relation as a Boolean formula over variables $x$ and
$x'$, where the primed variables $x'$ represent the next-state value of the
variable $x$.  A~\emph{trace} $\gamma: s_0,s_1,...$ is an infinite sequence
of states such that $s_0 \models I$, and for each $i \geq 0$, $(s_i,s_{i+1})
\models \delta$.  A state $s$ is reachable in $T$ if $\exists \gamma.\, s
\in \gamma$.  A safety property $P$ of $T$ is a boolean formula over the
variables $x$ of $T$, which asserts that certain states $s$ of $T$ cannot be
reached during the execution of $T$, often known as \emph{bad states}, given
as boolean formula $B(x)$.

\para{Problem Statement}
Given a state-space over $n$ Boolean variables, the problem is to
decide whether $T \models P$, that is, starting from initial state
$I(x)$, whether a state in $B(x)$ can be reached following only
transitions in $\delta(x,x')$.
%
