\section{Notation and Problem Statement}~\label{sec:notation}

A \textit{transition system} for bit-level hardware
modelling is a 3-tuple $T = \langle I,V,\delta \rangle$, where $V = \{v_1,\dots ,v_n\}$
is a finite set of \textit{state variables} ranging over the Booleans.
The propositional formula $I[V]$ over the variables in 
$V$ specifies the \textit{initial states} of the system. 
Corresponding to the variables in $V$ is a set of primed \textit{next-state} variables $V' = \{v_1', \dots, v_n'\}$.
The formula $\delta[V,V']$ represents the system's \textit{transition relation}; it specifies the relationship that the system realizes between current and next states at each discrete step of operation.

A \textit{state} $s$ of $T$ is an assignment of Boolean values to the variables in $V$.
If $F[V]$ is a formula over the variables $V$, we write $F(s)$ to say that $s$ makes the formula true.
So, for example, the initial states of a system are all the states $s$ for which $I(s)$. Similarly, if $R[V,V']$ is a relation over the variables $V$ and $V'$, we write $R(s_1,s_2)$ to say that the states $s_1$ and $s_2$ jointly satisfy this relation. Here, $s_2$ will, of course, be an assignment of values to the variables in $V'$. In general, when we mention a state $s$
it will be clear from contexts the underlying set of variables $s$ that assigns Boolean values to.

In the verification methods we investigate, it is common to analyse sequences of state transitions chained together.
We introduce sets of state variables $V_0, \dots, V_k$, where the variables in $V_0$ encode the initial state of the system and the variables in $V_i$, encode the state reached after $i>0$ transitions. With this encoding, the formula

\[ \underset{i \in [0,k-1]}{\bigwedge} \delta[V_i, V_{i+1}]\] 

\noindent can be used to characterize $k$ steps of the transition system.
We call a sequence of states $(s_j,\dots,s_k)$ a~\textit{trace}
if $\delta(s_i,s_{i{+}1})$ for $i=j,\dots,k-1$.  A trace is
said to be \textit{initialized} if $I(s_j)$.
  
A \textit{safety property} $P[V]$ is a predicate over the variables $V$ of $T$.  We will
say a state $s$ is \textit{good} with respect to this property if $P(s)$; otherwise it is \textit{bad}.
Finally, a property $P$ is \textit{inductive} if $P[V] \wedge \delta[V,V']$ implies $P[V']$.

\subsection{Problem Statement} 

We now define the \textit{safety verification} of a transition system~$T = \langle I,V,\delta\rangle$.
This is commonly known as the \textit{hardware model checking problem}.  
Given a safety property $P[V]$, the problem is to decide whether, starting from any initial state, a bad state can be reached following any sequence of transitions allowed by $\delta[V,V']$. In other words, is there an initialized trace of the system whose final state is bad?

