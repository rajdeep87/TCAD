\section{Notation and Problem Statement}~\label{sec:notation}

A Boolean \textit{transition system}, appropriate for bit-level hardware
modelling, is a 3-tuple $T = \langle I,X,\delta \rangle$, where $X = \{x_1,x_2,\dots ,x_n\}$
is a finite set of variables, each of which ranges over $\mathbb{B} =
\{\mathit{true}, \mathit{false}\}$.  
A \emph{state} $s$ of $T$ is an assignment of a Boolean value to each of the variables in $X$.  
The predicate $I(X)$ is a formula over
the variables in $X$ specifying the \emph{initial states} of the system.
The formula $\delta(X,X')$
represents the transition relation of the system over the variables in $X$ and
in $X' = \{x_1', x_2', \dots, x_n'\}$, where a primed variable $x'\in X'$ represent the next-state value of the corresponding variable $x\in X$.  


A~\emph{trace} $\gamma: s_0,s_1,\dots$ is an infinite sequence
of states such that \tmcmt{$s_0 \models I$, and for each $i \geq 0$, $(s_i,s_{i+1})
\models \delta$ TFM - This notation is not properly defined!}.  A state $s$ is reachable in $T$ if there exists a trace $\gamma$ in which $s$ occurs.  A safety property $P$ of $T$ is a boolean formula over the
variables $X$ of $T$. Such a property is interpreted  to asserts that certain states $s$ of $T$,
often known as \emph{bad states}, cannot be reached during the execution of $T$, given
as boolean formula $B(x)$. \tmcmt{I don't understand! What is B exactly?}


\para{Problem Statement}
We formally define the safety verification of a transition system~$T$,
commonly known as \emph{hardware model checking problem}.  
Given a state-space over $n$ Boolean variables, the problem is to
decide whether $T \models P$, that is, starting from initial state
$I(x)$, whether a state in $B(x)$ can be reached following only
transitions in $\delta(x,x')$.
%
