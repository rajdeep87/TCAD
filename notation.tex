\section{Notation and Problem Statement}~\label{sec:notation}

A Boolean \textit{transition system}, appropriate for bit-level hardware
modelling, is a 3-tuple $T = \langle I,X,\delta \rangle$, where $X = \{x_1,x_2,\dots ,x_n\}$
is a finite set of variables, each of which ranges over $\mathbb{B} = 
\{\mathit{true}, \mathit{false}\}$.  
%
Let $X$ and $X'$ are sets of present and next state variables respectively.  
% 
A \emph{state} $s$ of $T$ is an assignment of a Boolean value to variables $X$.  
The predicate $I(X)$ is a formula over
the variables in $X$ specifying the \emph{initial states} of the system.
The formula $\delta(X,X')$ represents the transition relation of the system 
over the variables in $X$ and $X'$. 
% 
We call a sequence of states $(s_j,\dots,s_k)$ a~\emph{trace} 
if $T(s_i,s_{i+1})$ is true for $i=j,\dots,k-1$.  A trace is 
initialized if the state $s_j$ satisfies the predicate $I(X)$.  
%  
A safety property $P(X)$ of $T$ is a predicate over the
variables $X$ of $T$.  We will say a state is \emph{good}, 
denoted by $G(X)$ (respectively \emph{bad}, $B(X)$) if it 
satisfies (respectively does not satisfy) $P(X)$.  A property 
$P$ is \emph{inductive} with respect to $T$ if 
$P(X) \wedge T \rightarrow P(X')$.    
%

\para{Problem Statement}
We formally define the safety verification of a transition system~$T$,
commonly known as \emph{hardware model checking problem}.  
Given a state-space over $n$ Boolean variables, the problem is to
decide whether $T \models P$, that is, starting from initial state
$I(X)$, whether a bad state can be reached following only transitions 
in $\delta(X,X')$.
%
