\subsection{Synthesizing software netlist from RTL}\label{sec:v2c}
%
\begin{figure*}[t]
\scriptsize  
\centering
\begin{tabular}{|l|l|l|l|}
\hline
  Verilog & Formal Semantics & Synthesized Hardware & Software netlist \\
\hline
\begin{lstlisting}[mathescape=true,language=Verilog]
module top(clk, a);
input clk, a;
reg b,d,e; 
wire c,cond;
assign c = e ? 1'b0:d;
assign cond = a;
always @(posedge clk) 
 begin
  b<=a;
  if(cond && b)
   e<=b;
  else 
   e<=0;
  d<=c;
 end
endmodule
\end{lstlisting}
&
\begin{minipage}{4.2cm}
%\centering
\scalebox{.5}{\import{figures/}{semantics.pspdftex}}
\end{minipage}
&
\begin{minipage}{4.0cm}
\centering
\scalebox{.5}{\import{figures/}{ckt.pspdftex}}
\end{minipage}
&
\begin{lstlisting}[mathescape=true,language=C]
struct state_elements_top {
 unsigned int b, d, e; };
struct state_elements_top  u1;
void top(_Bool clk, unsigned a) {
  _Bool c,cond;
  _Bool b_old=u1.b;
  cond = a;
  c = (u1.e)?0:u1.d;
  u1.b = a;
  if(cond && b_old)
    u1.e = b_old;
  else
    u1.e = 0;
  u1.d = c;  
}
\end{lstlisting}
\\
\hline
\end{tabular}
\caption{Circuit to Software}
\label{ex1}
\end{figure*}
%
%===============================================================================
\subsection{Software Netlist}
%===============================================================================
A hardware circuits in Verilog RTL is automatically synthesized 
into a word level C program, called a \textit{software 
netlist}~\cite{mtk2016,mskm2016} using the tool {\em v2c}~\cite{mtk2016}. 
{\em v2c} follows synthesis semantics and not simulation semantics.  The 
generated code is cycle-accurate as well as bit-precise. {\em v2c} 
automatically translates concurrent statements in Verilog such as 
procedural blocks and continuous assignments to a semantically 
equivalent C program. 
%
\begin{definition} 
A \textit{software netlist} $\mathcal{SN}$ is defined as follows:
%
\[ 
\begin{array}[t]{@{}lll}
\mathcal{SN} & {:}{:}{=} & \mathit{Modules} \\
\mathit{Modules} & {:}{:}{=} & \epsilon \mid \mathit{moduleName(Var_1,\dots,Var_n)} \{ Asgn \} \;\; Modules \\
\mathit{Asgn} &  {:}{:}{=} & \mathit{CAsgn} \mid \mathit{SAsgn} | moduleName(Var_1,\dots,Var_n)\\
\mathit{CAsgn} & {:}{:}{=} & (V_c := Bv) \mid (V_c := \mathit{Bool}), \quad V_c \in \mathit{Comb} \uplus \mathit{Out} \\
\mathit{SAsgn} & {:}{:}{=} & (V_s := Bv) \mid (V_s := \mathit{Bool}), \quad V_s \in \mathit{Seq} \\
\mathit{Bv} &  {:}{:}{=} & bv_{const} \mid bv_{var} \mid
	\mathit{ITE}(Bool, Bv, Bv) \mid \\
	& & bv_{op}(bv_1, \ldots, bv_n), bv_i \in Bv, \\ 
& & bv_{op} \in \{n-\text{ary operators over bitvectors} \} \\
\mathit{Bool} & {:}{:}{=} & \mathit{true} \mid \mathit{false} \mid \neg{Bool} \mid Bool \land Bool \mid 
Bool \lor Bool \mid \\ 
& & Bv\; relop\; Bv, relop \in \{<,\leq,=,>,\geq\} \\
\end{array}
\]
%
Here, $\mathit{In}$, $\mathit{Out}$, $\mathit{Seq}$, $\mathit{Comb}$
are programs variables that model \emph{input}, \emph{output}, 
\emph{sequential/state-holding} and \emph{combinational/stateless} 
signals in the circuit.  $Asgn$ is a finite set of assignments 
to $Out$, $Seq$ and $Comb$. 
\end{definition} 
%
A state transition in HW can be described by a set of register updates,
defined by the next-state function, and assignment of non-deterministic 
values to external inputs. The state transition in the software netlist 
corresponds to updates to the variables in $Seq$ and explicit assignment 
of non-deterministic values to the input variables, $In$.

\subsection{Structure of Software netlist:} The pseudo-code of the software 
netlist model generated using \emph{v2c} is shown in figure~\ref{figure:structure}.
%
\begin{figure}[t]
\captionsetup{justification=justified}
\scriptsize
\begin{tabular}{l}
\hline
 Pseudo-code for software-netlist model \\
\hline
\begin{lstlisting}[mathescape=true,language=C]
// parameter definition
// macro definition
struct state_elements_design
  // declare all state-holding elements
  // of the current module 
};
struct state_elements_design sdesign;
int initial_block() { //initialization of nets }
// Input are passed by value and output by reference
int design (data_type input, data_type *output) {

  // shadow variable declaration
  declare shadow variables for 
  non-blocking assignments to 
  the register elements  
  
  // continuous assignments
  Place all continuous statements which 
  are only dependant on input
 
  // always block 
  Place the always block respecting
  intra-modular dependencies
  // procedural statements are bit-precise

  // continuous assignments
  Place all continuous statements which 
  are updated by signals that are 
  driven from the always block

  // Module instantiations 
  Place all module instances 
  with proper mapping of 
  input and output ports

} // end of design module

int main() {
  // local varaibles 
  declare all local variables 
  which are passed to the design 
  initial_block(); // call to initial block
  // check if the design is sequential.
  // if so, then put a  while(1) wrapper
  while(1) {
   // nondeterministic assignments
   nondeterministically assign inputs values  
   // call the design
   design(input, &output);
  }
 
  // Assertions 
  Place the C assertions here 
} //end main 
\end{lstlisting}
\\
\hline
\end{tabular}
\caption{Skeleton of the software-netlist model generated using \emph{v2c}}
\label{figure:structure}
\end{figure}
%
%%===============================================================================\
\para{Procedural Assignments:}
%%===============================================================================\
%
Procedural assignments are used within Verilog always and initial blocks and
are of two types: \emph{blocking} and \emph{non-blocking}.  Blocking
assignments are executed in sequential order.  However, the effect of
blocking assignments is visible immediately, whereas the effect of
non-blocking assignments is delayed until all events triggered are
processed.  This form of parallelism in procedural assignments are modeled
in \emph{v2c} by first storing the value of register variables in auxiliary
variables in the beginning of the clock cycle.  Each read access to the
register variables are then replaced by these auxiliary variables.  This
ensures that an assignment to a register variable do not influence
subsequent procedural assignments.  Figure~\ref{figure:block} illustrates
the translation of procedural assignments (given at the top) to the
equivalent C semantics (given at the bottom).
%
\begin{figure*}
\scriptsize
\center
\begin{tabular}{l|l|l}
\hline
Non-blocking assignment & Blocking assignment & Continuous assignment \\
\hline
\begin{lstlisting}[mathescape=true,language=Verilog]
reg [7:0] x,y,z;
wire in = 1'b1;
always @(posedge clk) begin
 x <= in;
 y <= x;
 z <= y;
end
\end{lstlisting}
&
\begin{lstlisting}[mathescape=true,language=Verilog]
reg [7:0] x,y,z;
wire in = 1'b1;
always @(posedge clk) begin
 x = in;
 y = x;
 z = y;
end
\end{lstlisting}
&
\begin{lstlisting}[mathescape=true,language=Verilog]
wire in;
reg a,b,t;
wire a = in;
wire c = b; wire d = c; 
always @(posedge clk) begin
 b <= a;
 t <= b;
end 
\end{lstlisting}
\\
\hline 
\begin{lstlisting}[mathescape=true,language=C]
struct smain { 
unsigned char x,y,z; } sm;
unsigned char xs,ys,zs;
 _Bool in = 1;
// save register variables
 xs=sm.x;ys=sm.y;zs=sm.z;
// update register variables
 sm.x = in;
 sm.y = xs;
 sm.z = ys;
\end{lstlisting}
&
\begin{lstlisting}[mathescape=true,language=C]
struct smain {
unsigned char x,y,z;}sm;
 _Bool in = 1;
// clocked block
 sm.x = in;
 sm.y = sm.x;
 sm.z = sm.y;
\end{lstlisting}
&
\begin{lstlisting}[mathescape=true,language=C]
struct smain {
_Bool a,b,t; } sm;
_Bool in,c,d,as,bs,cs,ds,ts;
sm.a = in;//continuous assign
// save register variables
as=sm.a;bs=sm.b;ts=sm.t;
// clocked block
sm.b = as; sm.t = bs;
// continuous assignment 
c = sm.b; d = c;
\end{lstlisting}
\\
\hline
\end{tabular}
\caption{Tanslation of non-blocking, blocking and continuous assignments}
\label{figure:block}
\end{figure*}

%%===============================================================================\
\para{Bit-precise code generation:}
%%===============================================================================\
\emph{v2c} generates a bit-precise software netlist model in~C.  The tool
automatically handles complex bit-level operators in Verilog like
bit-select or part-select operators from a vector, concatenation operators,
reduction OR and other operators.  \emph{v2c} retains the word-level structure 
of the Verilog RTL and generates vectored expressions. 
Figure~\ref{figure:bit} shows Verilog code (at the top) and the generated C
expressions (at the bottom), which are combinations of bit-wise and
arithmetic operators like bit-wise OR, AND, multiplication, subtraction,
shifts and other C operators.

\begin{figure*}[htbp]
\scriptsize
\center
\begin{tabular}{l|l|l}
\hline
Bit-select & Part-select (SystemVerilog) & Concatenation \\
\hline
\begin{lstlisting}[mathescape=true,language=Verilog]
wire [7:0] in1,in2;
reg [7:0] out1,out2;
out1[7:5] = in1[4:2];
out2[6] = in2[4];
\end{lstlisting}
&
\begin{lstlisting}[mathescape=true,language=Verilog]
reg [31:0] in, out;
for(i=0;i<=3;i++) begin
out[8*i +: 8]=in[8*i +: 8];
end
\end{lstlisting}
&
\begin{lstlisting}[mathescape=true,language=Verilog]
wire [7:0] in1, in2;
reg [9:0] out;
out = {in2[5:2],in1[6:1]};
\end{lstlisting}
\\
\hline
\begin{lstlisting}[mathescape=true,language=C]
unsigned char in1,in2;
struct smain { 
 unsigned char out1,out2; } sm;
sm.out1 = sm.out1 & 0x1f | 
(((in1 & 0x1c)>>2)<<5);
sm.out2 = (sm.out2 & 0xbf)| 
(((in2 & 0x10)>>4)<<6); 
\end{lstlisting}
&
\begin{lstlisting}[mathescape=true,language=C]
struct smain {
 unsigned int in,out; } sm;
for(i=0;i<=3;i++) {
 x=8*i+(8-1); y=8*i;
 sm.out=(sm.out&!(2^31-2^y))
 |(sm.in&(2^31-2^y)); }
\end{lstlisting}
&
\begin{lstlisting}[mathescape=true,language=C]
unsigned char in1,in2;
struct smain { 
 unsigned char out; } sm;
sm.out = (((in2 >> 2)
 & 0xF) << 6)|
 ((in1 >> 1) & 0x3F);
\end{lstlisting}
\\
\hline
\end{tabular}
\caption{Handling Bit-select, part-select from vectors and concatenation operator}
\label{figure:bit}
\end{figure*}
%
%%===============================================================================\
\subsection{Dependency Analysis}
%%===============================================================================\
%
A hardware circuit specified in Verilog RTL may have two types of dependencies -- 
1) \emph{Intra-modular dependency} and b) \emph{Inter-modular dependency}. 
%
\subsubsection{Intra-Modular Dependency Analysis}
%
The intra-modular dependencies may occur due to the communication between
combinational blocks (continuous assignments) and sequential or clocked 
procedural blocks.  The following three cases summarizes the various sources of 
intra-modular dependencies in Verilog and provides the equivalent translation 
to a software netlist model.  Other forms of intra-modular dependencies are 
simply variants of these and can be handled appropriately.
Figure~\ref{dp1}, figure~\ref{dp2} and figure~\ref{dp3} graphically 
illustrate the intra-modular dependencies
where box denote the input, a bold circle denote a wire and a normal circle
denote a latch, bold edges denote the dependencies between two wires or a latch 
and a wire with the arrow pointing towards the wire and normal edges denote the 
dependencies between two latches or a wire and a latch with the arrow pointing
towards the latch. The dotted arrows denote the dependencies with an input.
\\
%
\noindent \textbf{Scenario A}  A wire, say $x$,  assigned in a continuous assignment 
statement, say $A$, appears in the right-hand side of another continuous assignment 
statement, say $B$.  This is illustrated in Figure~\ref{dp1}.\\

\noindent \textbf{Translation A} The variable assignment $A$ is placed before the other assignment $B$ 
which reads $x$.  \\

\noindent \textbf{Scenario B} A wire, say $x$, appearing in the right-hand side of a 
continuous assignment, say $A$, is driven by an always block.  This is illustrated in 
Figure~\ref{dp2}.\\

\noindent \textbf{Translation B} This gives an ordering where the continuous assignment is 
placed after the always block to capture the updated value of $x$. \\

\noindent \textbf{Scenario C} A latch, say $x$, appearing in procedural block, 
say $A$, is assigned directly by the input signal and $x$ is then read inside 
another procedural block. This is illustrated in Figure~\ref{dp3}.\\

\noindent \textbf{Translation C} The assignment to $x$ is placed before the second 
procedural block that reads $x$.\\

%
%\textbf{Dependencies among combinational assignments}
\begin{figure}
\scriptsize  
\centering
\begin{tabular}{l|l|l}
\hline
 Verilog & Dataflow Graph & Software netlist \\
\hline
\begin{lstlisting}[mathescape=true,language=Verilog]
module main();
wire x;
wire [1:0] y;
assign x=1'b1;
assign y=x+1'b1;
\end{lstlisting}
&
\begin{minipage}{2.0cm}
\centering
\scalebox{.5}{\import{figures/}{dp1.pspdftex}}
\end{minipage}
&
\begin{lstlisting}[mathescape=true,language=C]
int main() {
 bool x;
 unsigned char y;
 x=1;
 y=(x+1)&0x3;
 assert(y==2);
}
\end{lstlisting}
\\
\hline
\end{tabular}
\caption{Dependencies between combinational elements}
\label{dp1}
\end{figure}

%
%\textbf{A latch fed by the output of a combinational gate}
\begin{figure}
\scriptsize  
\begin{tabular}{l|l|l}
\hline
  Verilog & Dataflow Graph & Software netlist \\
\hline
\begin{lstlisting}[mathescape=true,language=Verilog]
module M(in1, in2, 
      out1, out2);
input in1, in2;
output reg out1,
output reg out2;
wire t1;
assign t1=out1;
always @(in1) begin
out1 <= in1;
end
always@(t1) begin
out2 <= t1;
end
endmodule
\end{lstlisting}
&
\begin{minipage}{1.8cm}
\centering
\scalebox{.5}{\import{figures/}{dp2.pspdftex}}
\end{minipage}
&
\begin{lstlisting}[mathescape=true,language=C]
int M(
bool in1,bool in2, 
bool *out1,
bool *out2) 
{
 bool t1;
 *out1=in1;
 t1=*out1;
 *out2=t1;
}
\end{lstlisting}
\\
\hline
\end{tabular}
\caption{Dependencies between latches and combinational logic}
\label{dp2}
\end{figure}


%\textbf{A latch fed by the output of another latch}
%
\begin{figure}
\scriptsize
\begin{tabular}{l|l|l}
\hline
  Verilog & Dataflow Graph & Software netlist \\
\hline
\begin{lstlisting}[mathescape=true,language=Verilog]
module M(in1, in2, 
      out1, out2);
input in1, in2;
output reg out1, out2;

always @(in1) begin
out1 <= in1;
end

always@(out1) begin
out2 <= out1;
end
endmodule
\end{lstlisting}
&
\begin{minipage}{2.0cm}
\centering
\scalebox{.5}{\import{figures/}{dp3.pspdftex}}
\end{minipage}
&
\begin{lstlisting}[mathescape=true,language=C]
int M(
bool in1,bool in2, 
bool *out1,*out2) {
 *out1=in1;
 *out2=*out1;
}
\end{lstlisting}
\\
\hline
\end{tabular}
\caption{Dependencies between latches}
\label{dp3}
\end{figure}
%
\Omit{
\textbf{A latch fed by combinational inputs}
%
\begin{figure}[t]
\scriptsize
\begin{tabular}{l|l|l}
\hline
  Verilog & Dataflow Graph & Software netlist \\
\hline
\begin{lstlisting}[mathescape=true,language=Verilog]
module M(in1, in2, out1, out2);
input in1, in2;
output reg out1, out2;

always @(in1) begin
out1 <= in1;
end

always@(in2) begin
out2 <= in2;
end
endmodule
\end{lstlisting}
&
\begin{minipage}{3.0cm}
\centering
\scalebox{.5}{\import{chapter3/figures/}{dp1.pspdftex}}
\end{minipage}
&
\begin{lstlisting}[mathescape=true,language=C]
int M(bool in1, bool in2, 
      bool *out1, *out2) {
 *out1=in1;
 *out2=in2;
}
\end{lstlisting}
\\
\hline
\end{tabular}
\caption{Dependencies between latch and combinational inputs}
\label{dp4}
\end{figure}
}
%
\Omit{For designs with inter-modular combinational paths or combinational loops,
the combinational signals (wire variables) may settle after several
executions before the next clock cycle.  The combinational exchanges between
modules depends on the stability condition for the combinational signals and
thus it is necessary to execute the combinational logic until the stability
condition is reached.  Determining such stability condition for large
circuits is hard.  An alternative way to handle combinational exchanges
between modules is by using assumptions over the signals that encode
combinational logic in the respective modules following synthesis semantics. 
An example using the latter approach is given at
\url{http://www.cprover.org/hardware/v2c/}.
}
%
\subsubsection{Inter-Modular Dependency Analysis}
%
Modules in Verilog communicate with each other through their input or output
ports. Most practical designs are modular in nature, where the top-level 
module delegates specific tasks to the sub-modules.
%through the input ports of sub-module and receive output from 
%the sub-module upon completion of the task.  
Modules and sub-modules executes \emph{in-tandem}, that is, a
module is \emph{not} blocked when it invokes a sub-module. 


An example of inter-modular communication can be illustrated
using the design in figure~\ref{figure:prop1}.  The top-level 
module \texttt{main} invokes the sub-module \texttt{initial} and \texttt{M}.  
The equivalent translation in C (shown in right of figure~\ref{figure:prop1} 
preserves the modular structure. 
%
Another example of inter-modular communication is a \emph{combinational feedback loop} 
where participating modules exchange combinational data until a stability condition is reached.  
%
