\section{Translating RTL to Software Netlists}\label{sec:v2c}
%
Verilog and C have many operators in common, but  Verilog also has 
additional hardware-oriented operators such as part-select, bit-select from vectors, concatenation, and reduction. Additionally, many Verilog statements have no direct analogue in C. Examples include
initial blocks, always blocks, generate statements, procedural assignments
(blocking, non-blocking) and continuous assignments. Verilog also supports the 4-valued data types 
reg, wire, and integer.  All these constructs, combined with 
parallelism, make translation challenging.

In view of these complexities, our translation is deliberately direct and transparent: we try to preserve the structure of the RTL and refrain from optimizations or abstractions. Fidelity of the resulting C code to the semantics of the original Verilog is prioritised---meaning that we rely on the downstream software analysis tools for efficiency, rather than obscure our translation with optimisations. 
%
\Omit{
\begin{figure*}[t]
\scriptsize  
\centering
%\begin{tabular}{|l|l|l|l|}
\begin{tabular}{|l|l|l|}
\hline
%  Verilog & Formal Semantics & Synthesized Hardware & Software netlist \\
\textbf{Verilog} & \textbf{Synthesized Hardware} & \textbf{Software Netlist} \\
\hline
\begin{lstlisting}[mathescape=true,language=Verilog]
module top(clk, a);
input clk, a;
reg b,d,e; 
wire c,cond;
assign c = e ? 1'b0:d;
assign cond = a;
always @(posedge clk) 
 begin
  b<=a;
  if(cond && b)
   e<=b;
  else 
   e<=0;
  d<=c;
 end
endmodule
\end{lstlisting}
&
\begin{minipage}{4.0cm}
\centering
\scalebox{.5}{\import{figures/}{ckt.pspdftex}}
\end{minipage}
&
\begin{lstlisting}[mathescape=true,language=C]
struct state_elements_top {
 unsigned int b, d, e; };
struct state_elements_top  u1;
void top(_Bool clk, unsigned a) {
  _Bool c,cond;
  _Bool b_old=u1.b;
  cond = a;
  c = (u1.e)?0:u1.d;
  u1.b = a;
  if(cond && b_old)
    u1.e = b_old;
  else
    u1.e = 0;
  u1.d = c;  
}
\end{lstlisting}\\
\hline
\end{tabular}
\caption{Example of Verilog Translated into C}
\label{ex1}
\end{figure*}
}
%
\subsection{Translation Flow and Software Netlists}

Our translation is implemented in a tool called V2C~\cite{mtk2016}.  
The front-end performs Verilog parsing, macro preprocessing, and
type-checking. The result is a type-annotated parse-tree, which is passed to the main translation phase. 
This performs rule-based translation following the Verilog module hierarchy, 
according to synthesis semantics. Rule-based translation produces vectored
assignments by mapping bit-operations to equivalent shift-and-mask operations in C, and
performs a global analysis to determine inter-module and intra-modular
dependencies.  The translation phase is followed by code-generation, 
where the intermediate vectored expressions and translated module items
are converted to code in ANSI-C. 

%
\subsection{Handling Verilog Module Items}
%
\subsubsection{Data Model} The data model in Verilog is significantly 
different from C.  Each bit of a C integer value can have 
only two states, namely 0 and 1.  The Verilog HDL
value sets consists of four states, namely 0, 1, $X$ and $Z$.  A value of 0
represents low voltage and value of 1 represents high voltage.  Further,
data values $X$ and $Z$ represent an unknown logic state and a 
high impedance value, respectively.  The simplest synthesis semantics 
for $X$ is treating it as a don't-care assignment, which allows the 
synthesis to choose a 0 or 1 to further improve logic minimization.
V2C treat $X$ and $Z$ states to be non-deterministic as there is no
corresponding C state. 

\subsubsection{Registers, wires, parameters and constants} Verilog supports structural 
data types called \emph{nets}, which are \emph{wire}
and \emph{reg}. The value of wire variables changes 
continuously as the input value changes.  Whereas, 
the \emph{reg} types hold their values until another value is 
assigned to them.  A structure containing all state holding
elements of a module is declared in C to store the register
variables. Wires are declared as local variables in C.  
Verilog Parameters are constants which 
are typically used to specify the width of variables,
for eg., \texttt{parameter identifier = constant\_expression;}.
Parameters are declared as constants in C.  
Verilog also allows the definition of the global constants 
using \texttt{`define} construct, for eg., \texttt{`define STATE 2'b00;},
which is same as the \#define preprocessor directives in C.

\subsubsection{Always and Initial blocks}
Always blocks are the concurrent statements which execute when a 
variable in the sensitivity list changes. The statements enclosed inside the 
always block within $begin \ldots end$ construct are executed in parallel
or sequentially depending on whether it is non-blocking or blocking statement 
respectively.   The behavior of an initial block is same as the always blocks, 
except that they are executed exactly once, before the execution of any 
always block. Figure~\ref{figure:always-init} shows the translation of always
and initial block (shown in top) to C (shown in bottom). All code-snippets are
partial and complete C code could not be accommodated due to space limit.
%
\begin{figure*}[htbp]
\scriptsize
\centering
\begin{tabular}{l|l}
\hline
Always and initial block & Multiple always and initial block\\
\hline
\begin{lstlisting}[mathescape=true,language=Verilog]
reg [7:0] x,y,z;
initial begin  x=0xFF;y=0;z=0; end
always @(posedge clk) begin
 y=x; z=y;
end
\end{lstlisting}
& 
\begin{lstlisting}[mathescape=true,language=Verilog]
reg [7:0] x,y,z;
initial begin  x=0xFF;y=0;z=0; end
always @(y) z=y+1;
always @(posedge clk) begin
 y=x;
end
\end{lstlisting}
\\
\hline 
\begin{lstlisting}[mathescape=true,language=C]
struct st {unsigned char x,y,z;} streg;
void initial() {
 streg.x=255;streg.y=0;streg.z=0;}
int clock_block() {
  streg.y=streg.x; 
  streg.z=streg.y;}
int main() {
 initial();
 while(1) clock_block();
}
\end{lstlisting}
&
\begin{lstlisting}[mathescape=true,language=C]
struct st {unsigned char x,y,z;} streg;
void initial() { 
 streg.x=255;streg.y=0;streg.z=0;}
int clock_block() {
  streg.y = streg.x; 
  streg.z = streg.y+1;
}
int main() {
 initial();
 while(1) clock_block();
} 
\end{lstlisting}
\\
\hline
\end{tabular}
\caption{Handling always and initial blocks}
\label{figure:always-init}
\end{figure*}

\subsubsection{Module Hierarchy with Input/Output Port}
The communication between modules takes place through 
ports, which are signals listed in the parameter list at the 
top of the module. Ports can be of type in, out, and inout. 
Figure~\ref{figure:module-hierarchy} shows an example of 
Verilog module hierarchy on the left and the translated 
code block in C on the right. The output ports are passed 
as reference to reflect the changes in the parent module.
The generated C code preserves the module hierarchy of 
the RTL and produce structurally identical code which 
often helps in mapping C-RTL operations for easier debugging. 

\begin{figure*}[htbp]
\scriptsize
\centering
\begin{tabular}{l|l}
\hline
Module Hierarchy with Input/Output Ports &  Code block in C \\
\hline
\begin{lstlisting}[mathescape=true,language=Verilog]
module top(in1, in2);
input [3:0] in1, in2;
wire [3:0] o1, o2;
and A1 (in1, in2, o1, o2);
and A2 (.c(o1),.d(o2),.a(o1),.b(in2));
endmodule
// Module Definition              
module and1(a, b, c, d);
input [3:0] a, b;
output [3:0] c, d;
reg [3:0] c;
always @(*) begin
 c = a & b;
end
assign d = 1;
endmodule
\end{lstlisting}
&
\begin{lstlisting}[mathescape=true,language=C]
struct state_elements_and {
 unsigned char c; };
struct state_elements_and sand;
void and(unsigned char a, unsigned char b, 
unsigned char *c, unsigned char *d) {
 *d = 1; sand.c = a & b;
}
void top(unsigned char in1, unsigned char in2) {
 unsigned char o1,o2;
 and(in1, in2, &o1, &o2);
 and(o1, in2, &o1, &o2);
}
void main() {
 unsigned char in1,in2;
 top(in1, in2);
}    
\end{lstlisting}
\\
\hline
\end{tabular}
\caption{Handling Module hierarchy with I/O ports}
\label{figure:module-hierarchy}
\end{figure*}

\subsubsection{Procedural Assignments}
%
Procedural assignments are used within Verilog 
procedures like always and initial blocks and are of 
two types: \emph{blocking} and \emph{non-blocking}.  
Blocking statements are executed in sequential order. However, blocking
assignments which are triggered from the same event execute
in parallel.  We model the parallelism using an auxiliary variable
for all the blocking assignments that are sensitive to the same event. 
On the other hand, the non-blocking statements always 
execute in parallel for the same clock. To model 
parallelism for non-blocking statements, we 
use shadow-variable based update technique. 
The value of register variables are stored in 
auxiliary variables in the beginning of the clock cycle.  
Each read access to the register variables is then replaced 
by these auxiliary variables. This ensures that 
an assignment to the register variable do not influence the 
subsequent assignments. Figure~\ref{figure:block} shows 
the translation of procedural assignments (shown in top) to 
the equivalent C semantics (shown in bottom).     

\begin{figure*}[htbp]
\scriptsize
\centering
\begin{tabular}{l|l|l}
\hline
Non-blocking assignment & Blocking assignment & Continuous assignment \\
\hline
\begin{lstlisting}[mathescape=true,language=Verilog]
reg [7:0] x,y,z;
wire in = 1'b1;
always @(posedge clk) begin
 x <= in;
 y <= x;
 z <= y;
end
\end{lstlisting}
&
\begin{lstlisting}[mathescape=true,language=Verilog]
reg [7:0] x,y,z;
wire in = 1'b1;
always @(posedge clk) begin
 x = in;
 y = x;
 z = y;
end
\end{lstlisting}
&
\begin{lstlisting}[mathescape=true,language=Verilog]
wire in;
reg a,b,t;
wire a = in;
wire c = b; wire d = c; 
always @(posedge clk) begin
 b <= a;
 t <= b;
end 
\end{lstlisting}
\\
\hline 
\begin{lstlisting}[mathescape=true,language=C]
struct smain { 
unsigned char x,y,z; } sm;
unsigned char xs,ys,zs;
 _Bool in = 1;
// save register variables
 xs=sm.x;ys=sm.y;zs=sm.z;
// update register variables
 sm.x = in;
 sm.y = xs;
 sm.z = ys;
\end{lstlisting}
&
\begin{lstlisting}[mathescape=true,language=C]
struct smain {
unsigned char x,y,z;}sm;
 _Bool in = 1;
// clocked block
 sm.x = in;
 sm.y = sm.x;
 sm.z = sm.y;
\end{lstlisting}
&
\begin{lstlisting}[mathescape=true,language=C]
struct smain {
_Bool a,b,t; } sm;
_Bool in,c,d,as,bs,cs,ds,ts;
sm.a = in;//continuous assign
// save register variables
as=sm.a;bs=sm.b;ts=sm.t;
// clocked block
sm.b = as; sm.t = bs;
// continuous assignment 
c = sm.b; d = c;
\end{lstlisting}
\\
\hline
\end{tabular}
\caption{Handling non-blocking, blocking and continuous assignments}
\label{figure:block}
\end{figure*}
%
\subsubsection{Continuous Assignment}
%
The continuous assignment is used to assign a value onto a 
wire in a module. Continuous assignments are concurrent statements 
which are immediately triggered when there is any change 
in the right-hand side inputs. The translation of continuous 
assignments are discussed next. 



\subsection{Bit-precise code generation}
V2C generate bit-precise software-netlist model in C.
The tool automatically handles complex bit-level operators in Verilog, 
like bit-select or part-select operators from a vector,
concatenation operators, reduction OR and other operators. V2C retains 
the word-level structure of the Verilog RTL and generate 
vectored expressions. Figure~\ref{figure:bit} shows 
Verilog code (at the top) and the generated C expressions 
(at the bottom) which are combinations of bitwise and 
arithmetic operators like bitwise OR, AND, multiplication, 
subtraction, shifts and other C operators.     
%
\Omit{
\subsubsection{Variable declaration} Variables of specific 
bit-width (register, wire) in Verilog is translated 
to the next largest native data-types in C such as 
char, short int, long, long long etc. The default 
variable declaration in C is \texttt{unsigned} except for 
the integer declaration in Verilog which is signed.
}
%
\begin{figure*}[htbp]
\scriptsize
\begin{tabular}{l|l|l}
\hline
Bit-select & Part-select (System Verilog) & Concatenation \\
\hline
\begin{lstlisting}[mathescape=true,language=Verilog]
wire [7:0] in1,in2;
reg [7:0] out1,out2;
out1[7:5] = in1[4:2];
out2[6] = in2[4];
\end{lstlisting}
&
\begin{lstlisting}[mathescape=true,language=Verilog]
reg [31:0] in, out;
for(i=0;i<=3;i++) begin
out[8*i +: 8]=in[8*i +: 8];
end
\end{lstlisting}
&
\begin{lstlisting}[mathescape=true,language=Verilog]
wire [7:0] in1, in2;
reg [9:0] out;
out = {in2[5:2],in1[6:1]};
\end{lstlisting}
\\
\hline
\begin{lstlisting}[mathescape=true,language=C]
unsigned char in1,in2;
struct smain { 
 unsigned char out1,out2;
} sm;
sm.out1 = sm.out1 & 0x1f | 
(((in1 & 0x1c)>>2)<<5);
sm.out2 = (sm.out2 & 0xbf)| 
(((in2 & 0x10)>>4)<<6); 
\end{lstlisting}
&
\begin{lstlisting}[mathescape=true,language=C]
struct smain {
 unsigned int in,out;
} sm;
for(i=0;i<=3;i++) {
 x=8*i+(8-1); y=8*i;
 sm.out=(sm.out&!(2^31-2^y))
 |(sm.in&(2^31-2^y)); 
}
\end{lstlisting}
&
\begin{lstlisting}[mathescape=true,language=C]
unsigned char in1,in2;
struct smain { 
 unsigned char out; 
} sm;
sm.out = (((in2 >> 2)
 & 0xF) << 6)|
 ((in1 >> 1) & 0x3F);
\end{lstlisting}
\\
\hline
\end{tabular}
\caption{Handling Bit-select, part-select from vectors and concatenation}
\label{figure:bit}
\end{figure*}
%
%%===============================================================================\
\subsection{Dependency Analysis}
%%===============================================================================\
%
A hardware circuit specified in Verilog RTL may have two types of dependencies -- 
1) \emph{Intra-modular dependency} and b) \emph{Inter-modular dependency}. 
%
\subsubsection{Intra-Modular Dependency Analysis}
%
The intra-modular dependencies may occur due to the communication between
combinational blocks (continuous assignments) and sequential or clocked 
procedural blocks.  We illustrate three different scenarios that summarize 
the various sources of intra-modular dependencies in Verilog and show 
how these dependencies are handled in a software netlist.  Other forms of 
intra-modular dependencies are simply variants of these three cases 
and can be handled appropriately.
%
Figure~\ref{dp1}--~\ref{dp3} graphically illustrate the intra-modular dependencies.  
%
Here, box denote an input, a bold circle denote a wire and a normal circle
denote a latch.  The bold edges denote the dependencies between two wires or a latch 
and a wire with the arrow pointing towards the wire and normal edges denote the 
dependencies between two latches or a wire and a latch with the arrow pointing
towards the latch. The dotted arrows denote the dependencies with an input. 
%
We describe the various scenarios and the corresponding translations below. 
Note that the code-snippets in Figure~\ref{dp1}--~\ref{dp3} are partial 
and are only used for illustrating the handling of intra-modular dependencies.
\\
%
\noindent \textbf{Scenario A}  A wire, say $x$,  assigned in a continuous assignment 
statement, say $A$, appears in the right-hand side of another continuous assignment 
statement, say $B$.  This is illustrated in Figure~\ref{dp1}.\\

\noindent \textbf{Translation A} The variable assignment $A$ is placed before the other assignment $B$ 
which reads $x$.  \\

\noindent \textbf{Scenario B} A wire, say $x$, appearing in the right-hand side of a 
continuous assignment, say $A$, is driven by an always block.  This is illustrated in 
Figure~\ref{dp2}.\\

\noindent \textbf{Translation B} This gives an ordering where the continuous assignment is 
placed after the always block to capture the updated value of $x$. \\

\noindent \textbf{Scenario C} A latch, say $x$, appearing in procedural block, 
say $A$, is assigned directly by the input signal and $x$ is then read inside 
another procedural block. This is illustrated in Figure~\ref{dp3}.\\

\noindent \textbf{Translation C} The assignment to $x$ is placed before the second 
procedural block that reads $x$.\\
%
%\textbf{Dependencies among combinational assignments}
\begin{figure}
\scriptsize  
\centering
\begin{tabular}{l|l|l}
\hline
 Verilog & Dataflow Graph & Software netlist \\
\hline
\begin{lstlisting}[mathescape=true,language=Verilog]
module main();
wire x;
wire [1:0] y;
assign x=1'b1;
assign y=x+1'b1;
\end{lstlisting}
&
\begin{minipage}{2.0cm}
\centering
\scalebox{.5}{\import{figures/}{dp1.pspdftex}}
\end{minipage}
&
\begin{lstlisting}[mathescape=true,language=C]
int main() {
 bool x;
 unsigned char y;
 x=1;
 y=(x+1)&0x3;
}
\end{lstlisting}
\\
\hline
\end{tabular}
\caption{Dependencies between combinational elements}
\label{dp1}
\end{figure}

%
%\textbf{A latch fed by the output of a combinational gate}
\begin{figure}
\scriptsize  
\begin{tabular}{l|l|l}
\hline
  Verilog & Dataflow Graph & Software netlist \\
\hline
\begin{lstlisting}[mathescape=true,language=Verilog]
module M(in1, in2, 
      out1, out2);
input in1, in2;
output reg out1,
output reg out2;
wire t1;
assign t1=out1;
always @(in1) begin
out1 <= in1;
end
always@(t1) begin
out2 <= t1;
end
endmodule
\end{lstlisting}
&
\begin{minipage}{1.8cm}
\centering
\scalebox{.5}{\import{figures/}{dp2.pspdftex}}
\end{minipage}
&
\begin{lstlisting}[mathescape=true,language=C]
int M(
bool in1,bool in2, 
bool *out1,
bool *out2) 
{
 bool t1;
 *out1=in1;
 t1=*out1;
 *out2=t1;
}
\end{lstlisting}
\\
\hline
\end{tabular}
\caption{Dependencies between latches and combinational logic}
\label{dp2}
\end{figure}


%\textbf{A latch fed by the output of another latch}
%
\begin{figure}
\scriptsize
\begin{tabular}{l|l|l}
\hline
  Verilog & Dataflow Graph & Software netlist \\
\hline
\begin{lstlisting}[mathescape=true,language=Verilog]
module M(in1, in2, 
      out1, out2);
input in1, in2;
output reg out1, out2;

always @(in1) begin
out1 <= in1;
end

always@(out1) begin
out2 <= out1;
end
endmodule
\end{lstlisting}
&
\begin{minipage}{2.0cm}
\centering
\scalebox{.5}{\import{figures/}{dp3.pspdftex}}
\end{minipage}
&
\begin{lstlisting}[mathescape=true,language=C]
int M(
bool in1,bool in2, 
bool *out1,*out2) {
 *out1=in1;
 *out2=*out1;
}
\end{lstlisting}
\\
\hline
\end{tabular}
\caption{Dependencies between latches}
\label{dp3}
\end{figure}
%
\Omit{
\textbf{A latch fed by combinational inputs}
%
\begin{figure}[t]
\scriptsize
\begin{tabular}{l|l|l}
\hline
  Verilog & Dataflow Graph & Software netlist \\
\hline
\begin{lstlisting}[mathescape=true,language=Verilog]
module M(in1, in2, out1, out2);
input in1, in2;
output reg out1, out2;

always @(in1) begin
out1 <= in1;
end

always@(in2) begin
out2 <= in2;
end
endmodule
\end{lstlisting}
&
\begin{minipage}{3.0cm}
\centering
\scalebox{.5}{\import{chapter3/figures/}{dp1.pspdftex}}
\end{minipage}
&
\begin{lstlisting}[mathescape=true,language=C]
int M(bool in1, bool in2, 
      bool *out1, *out2) {
 *out1=in1;
 *out2=in2;
}
\end{lstlisting}
\\
\hline
\end{tabular}
\caption{Dependencies between latch and combinational inputs}
\label{dp4}
\end{figure}
}
%
\Omit{For designs with inter-modular combinational paths or combinational loops,
the combinational signals (wire variables) may settle after several
executions before the next clock cycle.  The combinational exchanges between
modules depends on the stability condition for the combinational signals and
thus it is necessary to execute the combinational logic until the stability
condition is reached.  Determining such stability condition for large
circuits is hard.  An alternative way to handle combinational exchanges
between modules is by using assumptions over the signals that encode
combinational logic in the respective modules following synthesis semantics. 
An example using the latter approach is given at
\url{http://www.cprover.org/hardware/v2c/}.
}
%
\subsubsection{Inter-Modular Dependency Analysis}
%
Modules in Verilog communicate with each other through their input or output
ports. Most practical designs are modular in nature, where the top-level 
module delegates specific tasks to the sub-modules.
%through the input ports of sub-module and receive output from 
%the sub-module upon completion of the task.  
Modules and sub-modules executes \emph{in-tandem}, that is, a
module is \emph{not} blocked when it invokes a sub-module. 


An example of inter-modular communication can be illustrated
using the design in figure~\ref{figure:equivalence}.  The top-level 
module \texttt{main} invokes the submodule \texttt{M}.  
The equivalent translation in C (shown in right of figure~\ref{figure:equivalence}) 
preserves the modular structure. 
%
Another example of inter-modular communication is a \emph{combinational feedback loop} 
where participating modules exchange combinational data until a stability condition is reached.  
%
